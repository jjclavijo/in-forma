% Created 2018-04-06 vie 12:10
\documentclass[11pt]{article}
\usepackage[utf8]{inputenc}
\usepackage[T1]{fontenc}
\usepackage{fixltx2e}
\usepackage{graphicx}
\usepackage{longtable}
\usepackage{float}
\usepackage{wrapfig}
\usepackage{rotating}
\usepackage[normalem]{ulem}
\usepackage{amsmath}
\usepackage{textcomp}
\usepackage{marvosym}
\usepackage{wasysym}
\usepackage{amssymb}
\usepackage{hyperref}
\usepackage{bold-extra}
\tolerance=1000
\author{Javier J Clavijo}
\date{\today}
\title{pautas}
\hypersetup{
  pdfkeywords={},
  pdfsubject={},
  pdfcreator={Emacs 24.5.1 (Org mode 8.2.10)}}
\begin{document}

\maketitle
\tableofcontents

\section{Pautas para la elaboración de informes}
\label{sec-1}

Este documento pretende ser una guía sencilla de estilo sobre la elaboración de 
informes. Pedimos que la misma sea considerada a la hora de elaborar los 
trabajos prácticos de la materia, de manera que se facilite la comprensión tanto
del trabajo por parte del docente como de las correcciones posteriores por parte
del alumno.

\section{Sobre el estilo de escritura: El arte de la prosa.}
\label{sec-2}

El principal componente de todo informe es la escritura en prosa, a partir de
la que se desarrollan los conceptos que se quieren transmitir.

Definimos la prosa como "Forma de expresión habitual, oral o escrita, 
no sujeta a las reglas del verso." \footnote{\url{http://dle.rae.es/srv/fetch?id=UPg8xCx}, acepción 1.}. Sin embrago, se debe tener en cuenta que
la escritura en prosa no carece de reglas. Mas allá de las reglas ortográficas y
gramaticales, para que la transmisión de conceptos se desarrolle en forma eficiente
debemos aplicar lo que se conoce como reglas de estilo.

Antes de pasar a detallar las reglas de estilo que nos parecen importantes, hacemos 
notar que a la prosa también la definimos coloquialmente como "Demasía de palabras para 
decir cosas poco o nada importantes." \footnote{\url{http://dle.rae.es/srv/fetch?id=UPg8xCx}, acepción 4.}. Mencionamos esto porque es precisamente
lo que queremos evitar.

\subsection{Regla 1: Uso de los Tiempos Verbales}
\label{sec-2-1}

Durante la elaboración de un informe técnico, debemos conservar una coherencia en el
uso de los tiempos verbales. 

Los informes están relacionados en general a trabajos que ya fueron realizados, por lo 
tanto corresponde utilizar tiempos verbales pretéritos, conjugados en primera persona
del singular o del plural, dependiendo el caso.

En algunas circunstancias suele interrumpirse el discurso con alguna oración o sección
escrita utilizando el tiempo presente, para remarcar ideas o conceptos que no son particulares
al trabajo descrito sino que aclaran conceptos que subyacen detrás del mismo, por ejemplo:

"Realizamos la transformación de coordenadas utilizando una transformación de siete parámetros. 
esta transformación \textbf{se define} como [\ldots{}]"

Es de suma importancia mantener la coherencia a lo largo de todo el trabajo, repitiendo
el mismo de tipo de conjugación en todos los segmentos del texto que describan conceptos
de naturaleza similar.

Alternativamente, en caso de dividir el texto en secciones bien delimitadas, puede ser natural
utilizar tiempos verbales diferentes en secciones diferentes. 

A modo de ejemplo, supongamos un texto que tiene una sección de Metodología, una sección de 
Resultados y una sección de Conclusiones. Es natural en este caso que la primera se escriba 
en tiempo futuro, describiendo como \textbf{se desarrollará} el trabajo. Asimismo, la segunda sección
probablemente esté escrita en tiempo pretérito, explicando cual \textbf{fue} el resultado al llevar
a la practica la metodología descripta. Finalmente, las conclusiones pueden desarrollarse en 
tiempo presente, expresando las ideas que el autor \textbf{actualmente} tiene, luego de haber realizado
el trabajo.

\subsection{Regla 2: Uso de los Modos del discurso}
\label{sec-2-2}

Los modos del discurso son las formas en las que se construye una oración para expresar una idea.

Es importante que tengamos esto en cuenta a la hora de presentar un informe, porque cada modo
indica una relación particular entre el hablante y la idea que desarrolla.

Escribiremos utilizando modo indicativo aquellos conceptos que presentamos como hechos reales o
ideas ciertas.
Utilizaremos el modo subjuntivo en el caso de expresar ideas que son inciertas, hipotéticas o 
cuya verdad está condicionada a la verdad de premisas anteriores.
El modo imperativo suele quedar fuera de los informes técnicos, salvo en casos en que se lo utilice
a modo de énfasis. Por ejemplo, solemos indicar que determinada decisión metodológica se tomó "porque
el consenso es que el trabajo \textbf{debe} realizarse de esta manera si queremos asegurar su eficacia".

Esta regla está implícita en la manera en la que naturalmente se escribe o se habla, pero es importante
tenerla en cuenta en la etapa de revisión, antes de entregar el informe, para identificar si lo que se
está escribiendo es realmente lo que se quiere transmitir.

\subsection{Regla 3: Uso de la Voz Gramatical}
\label{sec-2-3}

Un factor sensible en la escritura de textos técnicos es el correcto uso de las voces gramaticales.
Por ejemplo, la oración precedente, y curiosamente esta misma oración y la que sigue a esta, están 
escritas utilizando la voz pasiva. Las oraciones están escritas utilizando la voz pasiva cuando el
sujeto del verbo no es quien ejecuta la acción que se está describiendo. 

Una oración escrita utilizando voz activa describe acciones realizadas por el mismo sujeto del verbo
de la oración. Ponemos como ejemplo de este caso la oración precedente, y esta misma, donde el sujeto
es el "nosotros" implícito en la conjugación del verbo.

Esta ampliamente extendido en la escritura de informes el uso de la voz pasiva, de modo que el énfasis
queda puesto en las acciones que fueron realizadas y no en el sujeto que las realizó.

\subsection{Regla 4: Un párrafo, Una idea}
\label{sec-2-4}

A la hora de escribir un texto de mediana extensión en prosa, es muy importante tener en cuenta las
divisiones que presenta naturalmente el mismo.

Una primera división se nos plantea en las oraciones que conforman el texto. Sobre esta, cabe decir 
que es mas claro el desarrollo de las ideas cuando las oraciones utilizadas son simples. Con esto
queremos remarcar que existen muchos casos en que una oración larga puede dividirse en varias 
oraciones más cortas y de estructura más simple. Un indicador natural de la complejidad en una 
oración es la cantidad de verbos y la cantidad de objetos que figuran en la misma. Cuando se tienen
frases largas, es posible separarlas individualizando cada verbo y su objeto en una oración independiente.

La siguiente división que se plantea, y tal vez la más importante, es la división en párrafos. Es de
vital importancia que cada párrafo plantee una idea y solo una idea. De esta manera, al realizar un
análisis del texto, podremos fácilmente identificar que partes de este son centrales y cuales no, a 
partir de la identificación de la idea que subyace a cada uno de ellas. Como ventaja adicional, una vez
que se completó la escritura de un texto, si hacemos el ejercicio de anotar en el margen de cada párrafo
cual es la idea que expresa, podremos detectar rápidamente si estamos repitiendo ideas a lo largo del
desarrollo, algo que dificulta la comprensión del texto por parte de terceros.

\section{Sobre las Figuras: cómo hacemos que una imagen complementaria se vea pertinente.}
\label{sec-3}

Un capitulo aparte merece la ubicación y puesta en contexto de las figuras y tablas dentro del texto 
del informe.

Cuando hablamos e figuras nos referimos a todo tipo de gráfico, esquema, dibujo o imagen que sea necesario
incluir para la completa comprensión de la información que se está transmitiendo.

\subsection{Figuras realmente pertinentes.}
\label{sec-3-1}

Sabemos que al incluir una imagen interrumpiremos de alguna manera el flujo de la lectura, ya sea visualmente,
porque la imagen apartará la atención del lector, o explícitamente, porque induciremos al lector a que 
antes de continuar la lectura observe determinada figura que se menciona en el texto.
Es por esto, que, aunque parezca contradictorio, la inclusión de figuras complementarias no siempre agrega
claridad al texto sino que muchas veces obstaculiza la transmisión de la información.

En consecuencia, a la hora de incluir una figura en un informe, debemos considerar que la misma sea 
realmente pertinente. Para esto, debemos considerar si la figura que estamos incluyendo simplifica 
la comprensión de los conceptos desarrollados, o por el contrario, si la idea que quiere reforzar
ya esta representada suficientemente en el texto. Veamos que una figura puede ser en si misma 
explicativa, agradable a la vista, e incluso sumamente informativa, pero si esa información distrae
la atención con respecto a ideas mas centrales, su inclusión puede ser perjudicial al conjunto.

\subsection{El contexto de la figura.}
\label{sec-3-2}

Cuando se incluye una figura o una tabla, existen varias maneras de ponerla en contexto. En el caso intuitivo,
presentaremos la figura inmediatamente antes de incluirla, es decir, el flujo del texto se vera interrumpido
por la figura, pero antes incluiremos una frase que la introduzca.
Si bien esta forma de utilizar una figura es muy común, pocas veces es la adecuada. Al mencionar explícitamente
en el texto a "la figura que sigue:" estamos forzando la interrupción de la lectura, y por tanto el desarrollo
conceptual. Esta interrupción solo debe darse en casos donde esté realmente justificada, como por ejemplo cuando
la información que presenta la figura sea parte indispensable de un concepto que se quiere transmitir, y no exista
forma de entender dicho concepto sin observarla.

Otra forma de introducir una figura, menos intuitiva al escribir pero mas natural para la lectura, es hacer mención
de ella en el texto en forma discreta, ya sea con una nota secundaria, o con una aclaración entre paréntesis. Esta
mención "al pasar" de una figura, no induce al lector a interrumpir la lectura, y en cambio lo informa de la existencia
de la información complementario. De esta manera, cuando el lector considere que es pertinente, interrumpirá la lectura
para observar la figura en cuestión.

\subsection{El pie de figura}
\label{sec-3-3}

La posibilidad de introducir figuras que no interrumpan el flujo de la lectura nos lleva a la necesaria inclusión de una
nota aclaratoria sobre la naturaleza de cada figura, en lo que se conoce como \textbf{pie de figura}.

En el pie de la figura incluiremos el nombre de la misma, que comúnmente es un numero correlativo, y una breve descripción
de su contenido. Esta información se coloca justo debajo o al margen de la figura, de manera que el lector pueda
independizar la lectura de las figuras de la lectura del texto, sin perder el contexto necesario
para comprender cada una por separado. En adición, nombrar las figuras nos permite referirnos a ellas en el texto
por su nombre, por ejemplo "En la Figura 1 se muestra \ldots{}".

\subsection{Ubicación en la página.}
\label{sec-3-4}

En caso de escribir el texto en forma tal que la ubicación de las figuras o las tablas no quede determinada por el mismo,
será preferible ubicar las mismas al principio o al final de la página, de modo que no interrumpan la lectura. 
En la mayor parte de los casos el tamaño de las figuras no necesita ser exageradamente grande, siendo posible ubicar
por ejemplo dos figuras una al lado de otra en el ancho de la página.

Suele recomendarse que la primer página de un trabajo no contenga figuras para evitar la interrupción de la 
lectura de las explicaciones introductorias, que en muchos casos son las que nos dan un panorama de qué podremos
esperar del trabajo y también una primera impresión a la luz de la cual mirar las figuras.

\section{Separación en Secciones}
\label{sec-4}

Más allá de la organización natural del texto que mencionamos anteriormente al hablar de los párrafos,
existe una organización que podemos llamar "\textbf{artificial}", que consiste en separar el texto por secciones a
las que separamos por medio de títulos o subtítulos.

En la gran mayoría de los casos es pertinente comenzar un trabajo con un pequeño resumen, de un párrafo único
por ejemplo, donde se describe cual es la idea general que corre detrás del trabajo que se va a leer. Esta sección
suele introducirse al inicio sin ningún titulo especial, pero con algún detalle tipográfico que la identifique.
Podemos asociar esta sección a lo que en un libro es el prologo o en su versión mas concisa un epígrafe.

En este punto, al referirnos a informes técnicos se nos abren dos posibilidades. El trabajo puede tener un hilo
conductor marcado alrededor de un único tema, o bien tratar de varios temas bien separados.

\subsection{Separación según el desarrollo metodológico.}
\label{sec-4-1}

Aplicado a los típicos trabajos prácticos de una materia, un ejemplo del primer tipo de informe se da cuando
debemos entregar un trabajo que consiste de un único ejercicio complejo o bien de varios ejercicios encadenados
que podrían interpretarse como uno solo. En este caso, si no hay una delimitación especial pedida por el docente, 
la división mas adecuada es la de "Materiales y Métodos", "Desarrollo", "Resultados", "Discusión" y "Conclusión".
donde las secciones de desarrollo y discusión pueden en ciertos casos fusionarse con otras si la comprensión 
no se ve perjudicada y en cambio la brevedad del trabajo beneficia al resultado.

En este esquema la sección de "Materiales y Métodos" detalla la fuente de los datos utilizados y la base teórica
sobre la que se desarrolla el trabajo. La sección de Desarrollo entrará en el detalle del trabajo realizando, 
explicando como se aplico puntualmente la metodología sobre los datos. La sección de Resultados presentará el
producto de los cálculos o experiencias realizadas, sin entrar en interpretaciones ni descripciones. En la sección
de Discusión se describen las posibles interpretaciones de los resultados, se destacan los puntos que deben
analizarse en profundidad y pueden presentarse cálculos o experiencias complementarias concisas
aclaren o complementen los resultados principales. Finalmente, la sección de Conclusiones retoma las lineas de 
discusión abiertas y realiza una argumentación breve, que permita cerrar el trabajo buscando las consecuencias
de las premisas que se expusieron durante la discusión.

\subsection{Separación por tema.}
\label{sec-4-2}

El segundo tipo de división mencionada, corresponde al caso en el que se trabaja sobre muchos temas 
distintos que no tienen una relación evidente entre sí, o bien, cuando se informa sobre un trabajo
que cuenta con múltiples ejercicios marcadamente distintos. 
En este caso, la división explícita sera la separación por \textbf{tema/ejercicio}, y toda otra división mas profunda
se encontrará en forma implícita en el texto.

Nos parece importante destacar, para terminar esta sección, que si bien la falta de separación en un trabajo
extenso puede hacer que este sea difícil de leer, también el exceso de divisiones forzadas en el texto contribuye
a interrumpir y obstaculizar el flujo de la lectura y el desarrollo de los conceptos.

\subsection{Divisiones débiles.}
\label{sec-4-3}

Mencionaremos otro tipo de división que se da en un texto, y que llamaremos aquí división débil.
Está división se da cuanto hacemos uso de recursos de formato para marcar una división a la que no
daremos entidad de sección individual sino que simplemente enfatizaremos la separación con un cambio
en la forma de la escritura.

En esta categoría, debemos considerar el uso de la sangría o el espaciado, que pueden marcar el inicio de un
nuevo tren de conceptos encadenados. También debemos considerar el uso de listas, variaciones en el tamaño de
los márgenes, secciones del texto enfatizados con letra \emph{itálica} o \textbf{negrita}, etc.

Todos estos recursos ayudan a atraer la atención del lector cuando se realiza un giro o cambio en el
desarrollo del texto, dándole a entender que hay un cambio mas o menos significativo en el contenido
pero sin llegar a constituirse un cambio completo de tema o una interrupción de la línea de argumentación.

\textsc{Un ultimo recurso dentro de esta categoría} es la de enfatizar con letras en versalita las primeras palabras
de un párrafo, marcando que constituye la introducción de una nueva idea.

\section{Notas al Pie y Citas Bibliográficas.}
\label{sec-5}

En caso de haber utilizado bibliografía de consulta para realizar un trabajo, esta situación debe estar
aclarada con una mención dentro del texto y un apartado nombrado convenientemente como "Bibliografía",
donde se lista todo el material consultado.

También, toda aclaración que no hace al trabajo, pero que es importante de hacer, como por ejemplo la fuente
de los datos utilizados o la dirección URL de acceso a un recurso, el nombre del software utilizado, etc.,
debe realizarse con un superíndice en el texto y la correspondiente nota al pie.

\section{Epílogo: Lista para Verificación de un escrito.}
\label{sec-6}

A continuación, y para tener una guía rápida a la hora de evaluar la estructura de un texto, introducimos
una lista de verificación que es conveniente revisar antes de presentar un trabajo.

\begin{itemize}
\item Carátula.
\begin{itemize}
\item $\square$ El trabajo tiene carátula.
\item $\square$ La carátula tiene datos del alumno.
\item $\square$ La carátula tiene datos del trabajo.
\item $\square$ la carátula tiene datos de los docentes.
\end{itemize}
\item Texto
\begin{itemize}
\item $\square$ Se conserva la coherencia en los tiempos verbales
\item $\square$ Se utiliza la voz pasiva cuando es necesario.
\item $\square$ Se evita el lenguaje coloquial.
\item $\square$ Se realizó corrección y revisión de la ortografía.
\item $\square$ Se utiliza estilo de texto justificado
\item $\square$ Cada párrafo expresa una idea.
\item $\square$ Se utilizó la sangría en primera línea al inicio de cada sección y para marcar divisiones importantes.
\item $\square$ Uso correcto de los signos de puntuación: comas, puntos, punto y coma, dos puntos, etc.
\end{itemize}
\item División
\begin{itemize}
\item Si el texto tiene Secciones.
\begin{itemize}
\item $\square$ Las secciones presentan un orden claro
\item $\square$ Los títulos o subtítulos están debidamente identificados
\item $\square$ Las divisiones no dificultan la lectura
\end{itemize}
\end{itemize}
\item Figuras
\begin{itemize}
\item $\square$ Las Figuras no interrumpen la linea de argumentación del texto.
\item $\square$ Cada figura está debidamente identificada y numerada.
\item $\square$ Los pies de figura contienen una descripción adecuada de cada figura.
\end{itemize}
\item Trabajos Prácticos
\begin{itemize}
\item $\square$ Se resolvieron todos los ejercicios.
\item $\square$ Se puede identificar a qué parte del enunciado responde cada sección.
\end{itemize}
\end{itemize}
% Emacs 24.5.1 (Org mode 8.2.10)
\end{document}
